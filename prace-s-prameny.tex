%%% Fiktivní kapitola s ukázkami citací

\chapter{Práce s literaturou}

Tato šablona předpokládá použití bibliografické databáze ve formátu \BibTeX\ z důvodu větší flexibility. Použití bibliografické databáze není nutnou podmínkou, lze si vystačit i se standardním prostředím \texttt{thebibliography}. V takovém případě je však zapotřebí provést zásahy do některých souborů, jak je uvedeno dále.

\section{Použití bibliografické databáze}

\begin{enumerate}
\item \textbf{Balíček \verb'biblatex', APA-7}\\
Šablona využivá pro zpracování bibliografické databáze nastavení přes balíček \verb|biblatex| a rovněž garantuje použití citačního standardu \textbf{APA-7}. Veškerá nastavení jsou uvedena v souboru \texttt{biblatex-setup.tex}.
\item\textbf{Změna názvu databáze}\\
V šabloně se předpokládá databáze uložená v souboru \texttt{literatura.bib}. Pokud se databáze jmenuje jinak, pak je nutné v souboru \texttt{biblatex-setup.tex} změnit hodnotu parametru příkazu \verb'\bibliography'.
\item\textbf{Změna citačního stylu}\\
Standardně se citace v textu uvádějí v kombinaci příjmení a roku (harvardský styl). Lze přepnout i na odkazy číslem změnou v souboru \texttt{biblatex-setup.tex}, kde se zruší komentářový znak v řádcích:
\begin{verbatim}
%  ,citestyle=numeric-comp
...
%\makeatletter
%\RequireBibliographyStyle{numeric}
%\makeatother
\end{verbatim}
\item \textbf{Využití populárního citačního manažeru Zotero}:
\begin{enumerate}
\item více informací -- \href{https://www.zotero.org/}{homepage}, \href{https://knihovna.vse.cz/citace/nastroje/zotero/}{informace z Knihovny VŠE}
\item Instalace -- \url{https://www.zotero.org/download/}
\item Instalace konektoru do prohlížeče -- Firefox, Chrome, Edge, Safari
\item Rozšíření Better BibTeX for Zotero -- \url{https://retorque.re/zotero-better-bibtex/}:
     \begin{enumerate}
     \item Stáhnout .xpi soubor
     \item A pak \texttt{Nástroje--Doplňky--Instal Add-on From File}
     \end{enumerate}
\item \href{https://formadoct.doctorat-bretagneloire.fr/zotero_workshop/latex}{Zotero workshop aneb Zotero\&{}\LaTeX\ krok po kroku}
\end{enumerate}
\end{enumerate}


\section{Použití prostředí \texttt{thebibliography}}
\begin{enumerate}
\item V souboru \texttt{makra.tex} vymazat na počátku tyto řádky:
\begin{verbatim}
%%% Nastavení pro použití samostatné bibliografické databáze.
%%% Settings for using a separate bibliographic database.
\input biblatex-setup
\end{verbatim}
\item V souboru \texttt{literatura.tex} odstranit řádek s příkazem \verb'\printbibliography' a odstranit příznak komentáře v další části obsahující prostředí \texttt{thebibliography.}
\item Jednotlivé položky \verb\bibitem\ musí být sestaveny dle standardu APA-7. Návodné příklady jsou k dispozici třeba zde: \url{https://knihovna.vse.cz/citace/priklady/?norm=apa}.
\end{enumerate}


\section{Jak citovat v textu}
\begin{center}
\begin{tabularx}{\textwidth}{l@{~~$\longrightarrow$~~}X}
\verb|\parencite{Cermak2018}|&\parencite{Cermak2018}\\
\verb|\parencite{Hladik2018,Jasek2018}|&\parencite{Hladik2018,Jasek2018}\\
\verb|\parencite[kap. 3]{Pecakova2018}|&\parencite[kap. 3]{Pecakova2018}\\
\verb|\parencite{Furtuna2023}|&\parencite{Furtuna2023}\\
\end{tabularx}
\end{center}
